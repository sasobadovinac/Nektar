\section{Refinements}
This section explains how to define a local $p$-refinement in a specific region in a mesh. After the user defines the polynomial expansions to be used, one can also locally change the polynomial order in a specific region in a mesh. Firstly, we introduce the expansion entry \inltt{REFIDS} which specifies the reference ID for a local $p$-refinement. In order words, this is the ID that connects the composite to the local $p$-refinements to be performed. This expansion entry must be added in the list of \inltt{<E>} elements as follows
\begin{lstlisting}[style=XMLStyle]
<E COMPOSITE="C[0]" NUMMODES="3" FIELDS="u" TYPE="MODIFIED" REFIDS="0" />
\end{lstlisting}
Then, in the REFINEMENTS section under \inltt{NEKTAR} tag as shown below, the local p-refinements are defined.
\begin{lstlisting}[style=XMLStyle]
<REFINEMENTS>
   ...
</REFINEMENTS>
\end{lstlisting}

The refinements entries are the reference ID (\inltt{REF}) which must match the one determined in the list of elements, the radius and two coordinates that define a cylindrical surface in a three-dimensional space and the number of modes. The example below shows the entries when only  the expansion type is provided.  Note that the local $p$-refinements are set as a list of \inltt{<R>} refinement regions.
\begin{lstlisting}[style=XMLStyle]
<R REF="0" 
   RADIUS="0.1" 
   COORDINATE1="0.1,0.2,0.1" 
   COORDINATE2="0.5,1.0,0.8" 
   NUMMODES="5" />
\end{lstlisting}
Thus, the elements which the vertices lay within this cylindrical surface (region) are refined based on the number of modes provided.  Note that this definition is for a three-dimensional mesh. For a two-dimensional mesh, a parallelogram surface is defined based on the same entries. The radius entry, in two-dimension, gives the length (diameter of a circle) of one pair of parallel sides. In a one-dimensional problem, a line is created and the radius entry gives the extra length in both coordinates. One important aspect to notice of this functionality is that the mesh dimension must match the space dimension.

When the expasion basis is specified in details as a combination of one-dimensional bases in the list of elements under the \inltt{EXPANSIONS} tag as shown below. The user also defines the number of quadrature points as explained in the previous section.
\begin{lstlisting}[style=XMLStyle]
<E COMPOSITE="C[0]" 
   BASISTYPE="Modified_A,Modified_A,Modified_A" 
   NUMMODES="3,3,3" 
   POINTSTYPE="GaussLobattoLegendre,GaussLobattoLegendre,GaussLobattoLegendre" 
   NUMPOINTS="5,5,5" 
   FIELDS="u" 
   REFIDS="0" />
\end{lstlisting}

Thus, an additional entry must be provided in the list of refinement regions when a detailed description of the expasion basis is given. In this case, the number of quadrature points have to be also given as follows 
\begin{lstlisting}[style=XMLStyle]
<R REF="0" 
   RADIUS="0.1" 
   COORDINATE1="0.1,0.2,0.1" 
   COORDINATE2="0.5,1.0,0.8"  
   NUMMODES="5,5,5"  
   NUMPOINTS="7,7,7" />
\end{lstlisting}

The $p$-refinement capability also allows the user to define multiple reference IDs (refienement regions) for each composite (see below). In other words, one can change the polynomial order in many locations in a mesh for a specific composite. It should be noted that if the user defines a region which is outside of the corresponding composite, the mesh is not going to be refined in the specified region. 
\begin{lstlisting}[style=XMLStyle]
<E COMPOSITE="C[0]" NUMMODES="3" FIELDS="u" TYPE="MODIFIED" REFIDS="0,1,2" />
<E COMPOSITE="C[1]" NUMMODES="5" FIELDS="u" TYPE="MODIFIED" REFIDS="3,4" />
\end{lstlisting}

The local p-refinement is only supported by CG discretisation at the moment.
