%%%%%%%%%%%%%%%%%%%%%%%%%%%%%%%
\chapter{CompressibleFlowSolver: Solving the Compressible Navier-Stokes Equations}

In this chapter, we walk the reader through our 2D and 3D compressible Navier-Stokes Solver (CompressibleFlowSolver). 
\section{Fundamental Theory of CompressibleFlowSolver}
The governing equation systems include continuity equation, momentum equations and energy equation. Written in conservative form 
\begin{equation}\label{eq1}
  \frac{d \textbf{U}}{d t}+
  \frac{\partial \textbf{F}_{i}(U)}{\partial x_{i}}+
  \frac{\partial \textbf{G}_{i}(U,\nabla U)}{\partial x_{i}}
  =0
\end{equation}
where $\textbf{U}=[\rho,\rho u_{1} \hdots \rho u_{d},E]^{T}$ is the vector of conservative variables.

And the inviscid flux $\textbf{F}$ in $i^{th}$ direction is
\begin{equation}
  \textbf{F}_{i}=
\begin{bmatrix}
  \rho u_{i}\\
  \rho u_{1}u_{i}+\delta_{1,i}p\\
  \vdots\\
  \rho u_{d}u_{i}+\delta_{d,i}p\\
  u_{i}(E+p)
\end{bmatrix}
\end{equation}
where total energy $E=\rho (c_{v}T+u_{i}u_{i}/2)$, pressure $p=(\gamma-1)(E-\rho u_{i}u_{i}/2)$

The viscous flux $\textbf{G}$ in $i^{th}$ direction is
\begin{equation}\label{eq7}
  \textbf{G}_{i}=
\begin{bmatrix}
  0\\
  -\tau_{i1}\\
  \vdots\\
  -\tau_{id}\\
  q_{i}-\sum\limits_{j=1}^{d}{u_{j}\tau_{ij}}
\end{bmatrix}
\end{equation}
where the viscous tensor $\tau_{ij}=\mu(\frac{\partial u_{i}}{\partial x_{j}}+\frac{\partial u_{j}}{\partial x_{i}}-\frac{2}{3}\frac{\partial u_{i}}{\partial x_{i}}\delta_{ij})$ and heat flux $q_{i}=-\kappa \frac{\partial T_{i}}{\partial x_{i}}$

In Nektar++, after the spatial discretization utilizing dicountinous Galerkin formulation, multiplied by a test function $\phi$, integrating inside an element $K\in \Omega_{h}$, we obtain Eequation \eqref{eq4}, , where $\Omega_{h}$ is the total computational region
\begin{equation}\label{eq4}
\int_{K}{\frac{d \textbf{U}^{e}}{d t}\phi^{e} dx}+I_{inv}+I_{vis}=0
\end{equation}
where
\begin{equation}\label{eq2}
I_{inv}={\int_{\partial K}{\sum\limits_{i=1}^{d}{\widetilde{\textbf{F}^{e}_{i}}(U)n_{i}\phi^{e}}}dS}-
{\int_{K}{\sum\limits_{i=1}^{d}{\textbf{F}^{e}_{i}(U)\frac{\partial \phi^{e}}{\partial x_{i}}dx}}}
\end{equation}
\begin{equation}\label{eq3}
I_{vis}= {\int_{\partial K}{\sum\limits_{i=1}^{d}{\widetilde{\textbf{G}}^{e}_{i}(U,\nabla U)n_{i}\phi^{e}}}dS}-
{\int_{K}{\sum\limits_{i=1}^{d}{\textbf{G}^{e}_{i}(U,\nabla U)\frac{\partial \phi^{e}}{\partial x_{i}}dx}}}
\end{equation}
are the inviscid and viscus fluxes.

Different flux terms such as Equation \eqref{eq2} and \eqref{eq3} and source terms are treated separately and transferred through member variables as shown in the diagram \ref{fig5}.
Various advection flux $\widetilde{\textbf{F}}$ and diffusion flux $\widetilde{\textbf{G}}(U,\nabla U)$ are supported in Nektar++ CompressibleSolver. And each specific type of flux such as LDG (DiffusionLDGNS) and Interior Penalty (DiffusionIP) diffusion fluxes inherit from their general flux class (Diffison).

\section{Advection flux $\textbf{F}_{i}$}
The following section takes AdvectionWeakDG.cpp as an example to demonstrate how the advection flux is calculated in Nektar++, where we have adopted the matrix notation from \cite{KaSh05} and a summary is proivded in Table \ref{table2}

Separate the advection flux from Equation \eqref{eq4}.
\begin{equation}
\int_{K}{\frac{d \textbf{U}^{e}}{d t}\phi^{e} dx}+I_{inv}=0
\end{equation}

Transforming Equation \eqref{eq4} into elemental matrix form, we obtain
\begin{equation}\label{eq5}
(B^{e})^{T}W^{e}B^{e}\frac{d \hat{\textbf{U}}^{e}}{d t}
-\sum\limits_{i=1}^{d}{(D^{e}_{x_{i}} B^{e})^{T}}W^{e}B^{e}\hat{\textbf{F}}^{e}_{i}+
b^{e}
=0
\end{equation}
where the integration of elemental trace flux is denoted by $b^{e}=\sum\limits_{i=1}^{d}{B^{e}\widetilde{\textbf{F}}^{e}_{i}}n_{i}$ and this further involves the evaluation of a Riemann flux $\widetilde{\textbf{F}_{i}}=\textbf{F}^{e}_{i}(U^{+},U^{-})$.

\begin {table}[H]
\caption {Item explanation} \label{table2} 
\begin{center}
\scalebox{0.9}[1.]{
\begin{tabular}{| c|c|c|c|}
  \hline      
$\phi^{e}$ & $\frac{\partial \phi^{e}}{\partial x_{i}}$& Diagonal weight matrix & Coefficients\\
  \hline
$B^{e}$ &$(D^{e}_{x_{i}} B^{e})^{T}$& $W^{e}$& $\hat{U}^{e}$,$\hat{F}^{e}$\\
  \hline
\end{tabular}
}
\end{center}
\end{table}

Now writing Equation \eqref{eq5} in semi-discretization form.
\begin{equation}\label{eq8}
\frac{d \hat{\textbf{U}}^{e}}{d t}
=((B^{e})^{T}W^{e}B^{e})^{-1}(\sum\limits_{i=1}^{d}{(D^{e}_{x_{i}} B^{e})^{T}}W^{e}B^{e}\hat{\textbf{F}}^{e}_{i}-
b^{e})
\end{equation}
where we note the elemental mass matrix is mass matrix $M^{e}=(B^{e})^{T}W^{e}B^{e}$.

Table \ref{table1} lists some functions that the process to calculate the advection term and provides a mapping from the Nektar++ compressible flow solver name to the mathematical operations they perofrm 
\begin {table}[h]
\caption {Table of variable and function mapping used in the compressible flow solver to their mathemaitcal operations} \label{table1} 
\begin{center}
\scalebox{0.9}[1.]{
\begin{tabular}{ | c | c|}
  \hline      
Variable/Function name & Physical meaning \\  
  \hline
  fluxVector  & Advection volume flux: $\textbf{F}_{i}$\\
  \hline
  numflux  & Advection (Riemann) numerical flux at trace: $\widetilde{\textbf{F}}_{i}$\\
   \hline
  IProductWRTDerivBase &  Volume flux integration: $\sum\limits_{i=1}^{d}{(D^{e}_{x_{i}} B^{e})^{T}}W^{e}B^{e}\hat{\textbf{F}}^{e}_{i}$ \\
   \hline
   AddTraceIntegral &  Add surface flux into volume flux interation: $+b^{e}$\\ 
  \hline
  MultiplyByElmImvMass& Multiply the inverse of mass matrix to get the flux coefficients $ \hat{\textbf{U}}$\\
 \hline
\end{tabular}
}
\end{center}
\end{table}


\section{Diffusion flux $\textbf{G}_{i}$}
The following section takes DiffusionLDGNS.cpp as an example to demonstrate how the diffusion flux is calculated in Nektar++. 

Separate the diffusion flux from Equation \eqref{eq4}.
\begin{equation}\label{eq6}
\int_{K}{\frac{d \textbf{U}^{e}}{d t}\phi^{e} dx}+I_{vis}=0
\end{equation}

As the resolution of diffusion flux is to solve a $2^{nd}$ PDE, it needs two equations to resolve the problem. LDG separates the resolution into two $1^{st}$ PDE.

From Equation \eqref{eq7}, diffusion derivatives $\textbf{G}_{i}(V,\nabla V)$ so that can reduce the computation using primitive variables $\textbf{V}=[u_{1}\hdots u_{d}, T]^{T}$. Therefore, the first step is to calculate the derivatives $\textbf{Q}_{i}=\frac{\partial V}{\partial x_{i}}$
\begin{equation}\label{eq9}
\int_{K}{\textbf{Q}^{e}_{i} \phi^{e} dx}=
{\int_{\partial K}{\widetilde{\textbf{V}}^{e}n_{i}\phi^{e}}dS}-
\int_{K}{\textbf{V}^{e}\frac{\partial \phi^{e}}{\partial x_{i}}dx}
\end{equation}

Transforming Equation \eqref{eq9} into elemental matrix form, we obtain
\begin{equation}\label{eq10}
\hat{\textbf{Q}}_{i}^{e}
=((B^{e})^{T}W^{e}B^{e})^{-1}
((D^{e}_{x_{i}} B^{e})^{T}W^{e}B^{e}\hat{\textbf{V}}^{e}-
b_{1,i}^{e})
\end{equation}
where the integration of elemental trace flux is denoted by $b_{1,i}^{e}=B^{e}\widetilde{\textbf{V}}n_{i}$, which involves the LDG numerical flux $\widetilde{\textbf{V}}=\textbf{V}^{e}_{i}(P^{+},P^{-})$.

When we get the diffusion derivatives $\textbf{G}_{i}$, the process to solve Equation \eqref{eq6} is simliar to the process of Equation \eqref{eq8}.
\begin{equation}\label{eq8}
\frac{d \hat{\textbf{U}}^{e}}{d t}
=((B^{e})^{T}W^{e}B^{e})^{-1}(\sum\limits_{i=1}^{d}{(D^{e}_{x_{i}} B^{e})^{T}}W^{e}B^{e}\hat{\textbf{G}}^{e}_{i}-
b_{2}^{e})
\end{equation}
where the integration of elemental trace flux is denoted by $b_{2}^{e}=\sum\limits_{i=1}^{d}{B^{e}\widetilde{\textbf{G}}^{e}_{i}}n_{i}$, which involves the LDG numerical flux $\widetilde{\textbf{G}_{i}}=\textbf{G}^{e}_{i}(G^{+},G^{-})$.

LDG numerical fluxes in Equations \eqref{eq9} and \eqref{eq8} satisfy

\begin{equation}
\tilde{\textbf{V}}=\{\textbf{V}\}+C_{12}\cdot [\textbf{V} n]
\end{equation}
where $C_{12}$ is a coefficient vector.
\begin{equation}
\tilde{\textbf{G}}=\{\textbf{G}\}-C_{12} [\textbf{G} \cdot n]
\end{equation}
where $[\quad]$ represents the jump while $\{\quad \}$ represents the average. In Nektar++, $C_{12}=\frac{1}{2}$, $C_{11}$=0


Table \ref{table3} lists some functions that the process to calculate the diffusion term and provides a mapping from the Nektar++ compressible flow solver name to the mathematical operations they perofrm.
\begin {table}[h]
\caption {Table of variable and function mapping used in the compressible flow solver to their mathemaitcal operations} \label{table3} 
\begin{center}
\scalebox{0.9}[1.]{
\begin{tabular}{ | c | c|}
\hline      
Variable/Function name & Physical meaning \\  
\hline
v\_NumericalFluxO1  &  Numerical flux $\tilde{\textbf{V}}^{e}$ \\
\hline
First IProductWRTDerivBase & Volume flux $(D^{e}_{x_{i}} B^{e})^{T}W^{e}B^{e}\hat{\textbf{V}}^{e}$\\
\hline
First AddTraceIntegral& Add numerical flux integration $+b_{1,i}^{e}$ \\
\hline
MultiplyByElmImvMass& Multiply the inverse of mass matrix to get the flux coefficients $\textbf{Q}$\\
\hline
m\_fluxVectorNS  & Diffusion volume flux: $\textbf{G}_{i}$\\
\hline
v\_NumericalFluxO2 & Diffusion numerical flux : $\widetilde{\textbf{G}}_{i}$\\
 \hline
Second IProductWRTDerivBase &  Volume flux integration: $\sum\limits_{i=1}^{d}{(D^{e}_{x_{i}} B^{e})^{T}}W^{e}B^{e}\hat{\textbf{G}}^{e}_{i}$ \\
 \hline
Second AddTraceIntegral &  Add surface flux into volume flux interation: $+b_{2}^{e}$\\ 
\hline
MultiplyByElmImvMass& Multiply the inverse of mass matrix to get the flux coefficients $\hat{\textbf{U}}$\\
\hline
\end{tabular}
}
\end{center}
\end{table}







\clearpage
\section{Data Structure of CompressibleFlowSolver}
   \begin{figure}
          \caption{CompressibleFlowSolver DataStructure}\label{fig5}
        \centering
        \begin{turn}{90}
        \includestandalone[width=1.2\linewidth]{DataStructure}
        \end{turn}
    \end{figure}

\clearpage
\section{Flow Chart of CompressibleFlowSolver}
   \begin{figure}
          \caption{CompressibleFlowSolver Main Flow Chart}
        \centering
      \begin{tikzpicture}[scale=0.2,node distance=1cm]
\node (A)
[rectangle,
rounded corners,
minimum width=8cm,
minimum height=1cm,
text centered,
draw=black,
fill=red!30]
{CompressibleFlowSolver$.$cpp};
\node (B_1)
[rectangle,
rounded corners,
minimum width=8cm,
minimum height=1cm,
text width=8cm,
text centered,
draw=black,
fill=blue!20,
below =of A,
xshift=0cm,
yshift=0cm]
{Initialize Objects\\eg. DriverStandard::v$\_$InitObject\\See Figure \ref{fig1}};
\draw[arrow](A)--(B_1);
\node (B_2)
[rectangle,
rounded corners,
minimum width=8cm,
minimum height=1cm,
text width=8cm,
text centered,
draw=black,
fill=blue!20,
below =of B_1,
xshift=0cm,
yshift=0cm]
{Execute Solver beginning from driver:\\eg. DriverStandard::v$\_$Execute};
\draw[arrow](B_1)--(B_2);
\node (C_1)
[rectangle,
rounded corners,
minimum width=8cm,
minimum height=1cm,
text width=8cm,
text centered,
draw=black,
fill=yellow!20,
right= of B_2,
xshift=0cm,
yshift=0cm]
{Initial conditions: \\m$\_$equ[0]-$>$DoInitialise\\ See Figure \ref{fig2}};
\draw[arrow](B_2.east)--(C_1);
\node (C_2)
[rectangle,
rounded corners,
minimum width=8cm,
minimum height=1cm,
text width=8cm,
text centered,
draw=black,
fill=yellow!20,
below = of C_1,
xshift=0cm,
yshift=0cm]
{Main execution of solver:\\m$\_$equ[0]-$>$DoSolve\\ See Figure \ref{fig3} and Figure \ref{fig4}};
\draw[arrow](B_2.east)--($(B_2.east)+(2cm,0)$)|-(C_2);
\node (C_3)
[rectangle,
rounded corners,
minimum width=8cm,
minimum height=1cm,
text width=8cm,
text centered,
draw=black,
fill=yellow!20,
below = of C_2,
xshift=0cm,
yshift=0cm]
{Output brief information:\\m$\_$equ[0]-$>$Output};
\draw[arrow](B_2.east)--($(B_2.east)+(2cm,0)$)|-(C_3);
\end{tikzpicture}
    \end{figure}

\clearpage
   \begin{figure}
          \caption{CompressibleFlowSolver InitObject}\label{fig1}
        \centering
        \includestandalone[width=1\linewidth]{FlowChart1}
    \end{figure}

\clearpage
   \begin{figure}
          \caption{CompressibleFlowSolver Initialize Conditions}\label{fig2}
        \centering
        \includestandalone[width=1\linewidth]{FlowChart2}
    \end{figure}

\clearpage
   \begin{figure}
          \caption{CompressibleFlowSolver Execute Advection}\label{fig3}
        \centering
        \includestandalone[width=0.5\linewidth]{FlowChart3}
   \end{figure}

\clearpage
   \begin{figure}
          \caption{CompressibleFlowSolver Execute Diffusion}\label{fig4}
        \centering
        \includestandalone[width=0.5\linewidth]{FlowChart4}
   \end{figure}


% \section{The Fundamentals Behind the Compressible Flow Solver}

% \subsection{Strong Form of the System}

% \subsection{Variational Form of the System}

% \section{The Fundamentals Behind the Data Structures}

% \subsection{Overview}

% \subsection{Equation System Description}

% \subsection{Advection Components}

% \subsection{Diffusion Components}


% \section{The Fundamental Flow Control of The Solver: Flow Chart}

% \subsection{Top-Down Perspective}

% \subsection{Evaluation of the Explicit Term}

% \subsection{Evaluation of the Implicit Term}

% \section{The Fundamental Algorithms}

% \paragraph{How Would I Change From External Energy Form to Internal Energy Form?}

% \paragraph{How Is De-Aliasing Applied to the Non-Linear Terms?}

