%%%%%%%%%%%%%%%%%%%%%%%%%%%%%%%
\chapter{CompressibleFlowSolver: Solving the Compressible Navier-Stokes Equations}

In this chapter, we walk the reader through our 2D and 3D compressible Navier-Stokes Solver (CompressibleFlowSolver). 
\section{Fundamental Theories of CompressibleFlowSolver}
The governing equation systems include continuity equation, momentum equations and energy equation. Write in conservative form 
\begin{equation}\label{eq1}
  \frac{\partial U}{\partial t}+
  \frac{\partial F_{i}(U)}{\partial x_{i}}
  =\frac{\partial G_{i}(U,\nabla U)}{\partial x_{i}}
\end{equation}
where $U=[\rho,\rho u_{1} \hdots \rho u_{d},E]^{T}$

Inviscid flux F in i direction is
\begin{equation}
  F_{i}=
\begin{bmatrix}
  \rho u_{i}\\
  \rho u_{1}u_{i}+\delta_{1,i}P\\
  \vdots\\
  \rho u_{d}u_{i}+\delta_{d,i}P\\
  u_{i}(E+P)
\end{bmatrix}
\end{equation}
where total energy $E=\rho (c_{v}T+u_{i}u_{i}/2)$, pressure $P=(\gamma-1)(E-\rho u_{i}u_{i}/2)$

Viscous flux G in i direction is
\begin{equation}
  G_{i}=
\begin{bmatrix}
  0\\
  \tau_{i1}\\
  \vdots\\
  \tau_{id}\\
  \sum\limits_{j=1}^{d}{u_{j}\tau_{ij}}-q_{i}
\end{bmatrix}
\end{equation}
where viscous term $\tau_{ij}=\mu(\frac{\partial u_{i}}{\partial x_{j}}+\frac{\partial u_{j}}{\partial x_{i}}-\frac{2}{3}\nabla \cdot\textbf{u}\delta_{ij})$ and heat flux $\textbf{q}=-\kappa \nabla T$

In Nektar++ current compressible flow solver, DG spatial discretization is in default. Multiply a test function $\phi$, integrate inside an element $K\in \Omega_{h}$ and sum over all the elements.
\begin{equation}
  \sum\limits_{K\in \Omega_{h}}{\int_{K}{\frac{\partial U}{\partial t}\phi dx}}+I_{inv}=I_{vis}
\end{equation}
where
\begin{equation}\label{eq2}
I_{inv}= \sum\limits_{K\in \Omega_{h}}{\int_{\partial K}{\sum\limits_{i=1}^{d}{\widetilde{F}_{i}(U)n_{i}\phi}}dS}-
\sum\limits_{K\in \Omega_{h}}{\int_{K}{\sum\limits_{i=1}^{d}{F_{i}(U)\frac{\partial \phi}{\partial x_{i}}dx}}}
\end{equation}
\begin{equation}\label{eq3}
I_{inv}= \sum\limits_{K\in \Omega_{h}}{\int_{\partial K}{\sum\limits_{i=1}^{d}{\widetilde{G}_{i}(U,\nabla U)n_{i}\phi}}dS}-
\sum\limits_{K\in \Omega_{h}}{\int_{K}{\sum\limits_{i=1}^{d}{G_{i}(U,\nabla U)\frac{\partial \phi}{\partial x_{i}}dx}}}
\end{equation}

Different flux terms such as Equation \eqref{eq2} and \eqref{eq3} and source terms are treated separated and transferred through member variables as the sketch \ref{fig1} shows.
Various advection flux $\widetilde{F}$ and diffsion flux $\widetilde{G}(U,\nabla U)$ are supported in Nektar++ CompressibleSolver. And each specific flux such as LDG diffusion flux (LDGNS) inherits from its general flux type (Diffison).

Usually inside one specific flux type, the codes includes the following simliar process to caculate the flux. Take AdvectionWeakDG.cpp as an example.
\begin {table}[!h]
\caption {AdvectionWeakDG.cpp} \label{table1} 
\begin{center}
\scalebox{0.9}[1.]{
\begin{tabular}{ | c | c|}
  \hline      
Variable/Function name & Physical meaning \\  
  \hline
  m$\_$fluxVector  & Advection volume flux: $F_{i}$\\
  \hline
  numflux  & Advection numerical flux at trace: $\widetilde{F}_{i}$\\
   \hline
  IProductWRTDerivBase &  Volume flux integration: $\sum\limits_{K\in \Omega_{h}}{\int_{K}{\sum\limits_{i=1}^{d}{F_{i}(U)\frac{\partial \phi}{\partial x_{i}}dx}}}$ \\
   \hline
   AddTraceIntegral &  Add Surface flux into volume flux interation: $+\sum\limits_{K\in \Omega_{h}}{\int_{\partial K}{\sum\limits_{i=1}^{d}{\widetilde{F}_{i}(U)n_{i}\phi}}dS}$\\ 
  \hline
  MultiplyByElmImvMass& Multiply the inverse of mass matrix to get the flux coefficients\\
 \hline
\end{tabular}
}
\end{center}
\end{table}
\clearpage
\section{Data Structure of CompressibleFlowSolver}
   \begin{figure}\label{fig1}
          \caption{CompressibleFlowSolver DataStructure}
        \centering
        \begin{turn}{90}
        \includestandalone[width=1.2\linewidth]{DataStructure}
        \end{turn}
    \end{figure}

\clearpage
\section{Flow Chart of CompressibleFlowSolver}
   \begin{figure}
          \caption{CompressibleFlowSolver Main Flow Chart}
        \centering
      \begin{tikzpicture}[scale=0.2,node distance=1cm]
\node (A)
[rectangle,
rounded corners,
minimum width=8cm,
minimum height=1cm,
text centered,
draw=black,
fill=red!30]
{CompressibleFlowSolver$.$cpp};
\node (B_1)
[rectangle,
rounded corners,
minimum width=8cm,
minimum height=1cm,
text width=8cm,
text centered,
draw=black,
fill=blue!20,
below =of A,
xshift=0cm,
yshift=0cm]
{Initialize Objects\\eg. DriverStandard::v$\_$InitObject\\See Figure \ref{fig1}};
\draw[arrow](A)--(B_1);
\node (B_2)
[rectangle,
rounded corners,
minimum width=8cm,
minimum height=1cm,
text width=8cm,
text centered,
draw=black,
fill=blue!20,
below =of B_1,
xshift=0cm,
yshift=0cm]
{Execute Solver beginning from driver:\\eg. DriverStandard::v$\_$Execute};
\draw[arrow](B_1)--(B_2);
\node (C_1)
[rectangle,
rounded corners,
minimum width=8cm,
minimum height=1cm,
text width=8cm,
text centered,
draw=black,
fill=yellow!20,
right= of B_2,
xshift=0cm,
yshift=0cm]
{Initial conditions: \\m$\_$equ[0]-$>$DoInitialise\\ See Figure \ref{fig2}};
\draw[arrow](B_2.east)--(C_1);
\node (C_2)
[rectangle,
rounded corners,
minimum width=8cm,
minimum height=1cm,
text width=8cm,
text centered,
draw=black,
fill=yellow!20,
below = of C_1,
xshift=0cm,
yshift=0cm]
{Main execution of solver:\\m$\_$equ[0]-$>$DoSolve\\ See Figure \ref{fig3} and Figure \ref{fig4}};
\draw[arrow](B_2.east)--($(B_2.east)+(2cm,0)$)|-(C_2);
\node (C_3)
[rectangle,
rounded corners,
minimum width=8cm,
minimum height=1cm,
text width=8cm,
text centered,
draw=black,
fill=yellow!20,
below = of C_2,
xshift=0cm,
yshift=0cm]
{Output brief information:\\m$\_$equ[0]-$>$Output};
\draw[arrow](B_2.east)--($(B_2.east)+(2cm,0)$)|-(C_3);
\end{tikzpicture}
    \end{figure}

\clearpage
   \begin{figure}
          \caption{CompressibleFlowSolver InitObject}\label{fig1}
        \centering
        \includestandalone[width=1\linewidth]{FlowChart1}
    \end{figure}

\clearpage
   \begin{figure}
          \caption{CompressibleFlowSolver Initialize Conditions}\label{fig2}
        \centering
        \includestandalone[width=1\linewidth]{FlowChart2}
    \end{figure}

\clearpage
   \begin{figure}
          \caption{CompressibleFlowSolver Execute Advection}\label{fig3}
        \centering
        \includestandalone[width=0.5\linewidth]{FlowChart3}
   \end{figure}

\clearpage
   \begin{figure}
          \caption{CompressibleFlowSolver Execute Diffusion}\label{fig4}
        \centering
        \includestandalone[width=0.5\linewidth]{FlowChart4}
   \end{figure}


% \section{The Fundamentals Behind the Compressible Flow Solver}

% \subsection{Strong Form of the System}

% \subsection{Variational Form of the System}

% \section{The Fundamentals Behind the Data Structures}

% \subsection{Overview}

% \subsection{Equation System Description}

% \subsection{Advection Components}

% \subsection{Diffusion Components}


% \section{The Fundamental Flow Control of The Solver: Flow Chart}

% \subsection{Top-Down Perspective}

% \subsection{Evaluation of the Explicit Term}

% \subsection{Evaluation of the Implicit Term}

% \section{The Fundamental Algorithms}

% \paragraph{How Would I Change From External Energy Form to Internal Energy Form?}

% \paragraph{How Is De-Aliasing Applied to the Non-Linear Terms?}

